\chapter*{Introduction}
This report was written as part of the course SDU-UAS-Intro 2015 and presents the work of group 1. 
The aim of the developed system is to enable an existing drone model to easily be configured and
programmed for performing offshore vision aided patrolling. 
The used equipment was a 3DR Iris+ drone with a Tarot T-2D Brushless gimbal mounted Go Pro HERO 3 camera.
This enabled the drone to get visual data from %restricted,
user-specified areas. %at some distance from the shore.
During its watch, if any human being is detected on this area, the drone should store the acquired images and send
them to the ground control station on the fly, to be analysed by the user and activate the rescue protocols.

\section*{Motivation}
Patrolling and ensuring safety at remote and humanly inaccessible areas such as the sea has 
been a difficult task for a long time, requiring expensive equipments and many man hours,
see \cite{Ref:Drone2}.
With the recent development of readily accessible drones, which are easy to use and fairly simple to operate,
a revolution within the area of life guarding seems within reach. 

Utilizing the drone's capability of operating and monitoring humanly inaccessible areas to control coast areas
could not only result in a more safe coast environment, but also a decreased cost in regards to life saving.
The use of drones in regards to emergencies and rescue missions has already showed promising results
\cite{Ref:Drone1}, \cite{Ref:Drone3}, \cite{Ref:Drone4}, \cite{Ref:DroneResearch1}, \cite{Ref:DroneResearch1},
but mostly using humanly controlled flights.

The vision is to explore the possibilities of making an easy to use patrolling system
and making a drone fly autonomously within a specified area.
Using computer vision to detect changes in the current scene or movement it should report back with images from the scene.
The images can then be processed by a professional. As this project is considered a research project the material presented
will be limited to the areas of autonomous flight in good flying conditions and computer vision.
\\ 
Using this approach, a minimum amount of man hours is needed, while the possibilities of autonomous drone flight will be utilized.
\\
To ensure efficiency, the group was divided into two subgroups: One for the Autonomous flight and one for the Human detection through computer vision.
The group work distribution and contributions to the report by each group member can be seen in \ref{app:work}.

\newpage
\section*{Abbreviations}
\begin{acronym}[AWGN]
\acro{UAS}{Unmanned Aerial System}
\acro{ROS}{Robot Operating System} 
\acro{UART}{Universal asynchronous receiver/transmitter} 
\acro{RTL}{Return to launch} 
\acro{RC}{Remote Control}
\acro{FCU}{Flight Controller Unit}
\acro{FOV}{Field of view}
\acro{ROI}{Region of interest}
\acro{fps}{Frames per second}

\end{acronym}