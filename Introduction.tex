\chapter*{Introduction}

The present report contains the description of the  project "Using the IRIS + quadcopter for lifeguarding" developed as a part of the course  SDU-UAS-Intro 2015. 
The aim of the developed system is to enable an existing drone model to be easily configured and programmed for performing offshore patrolling with human rescue purposes. 
The required equipment, a quadcopter with a mounted gimbal with a camera, can be thus used to fly over restricted, user-specified areas at some distance from the shore. 
During its watch, if any human being is detected on this area, the drone stores the acquired images and send them on the fly to the ground station to be analyzed by the user and activate the rescue protocols. 

\section*{Acronyms}
\begin{acronym}[AWGN]
\acro{UAS}{Unmanned Aerial Systems}
\acro{ROS}{Robotics Operation System} 
\acro{UART}{Universal asynchronous receiver/transmitter} 
\acro{RTL}{Return to land} 

\end{acronym}

\section*{Motivation}
Patrolling and ensuring safety at remote and humanly inaccessible areas such as the sea has for a long time been a difficult task requiring expensive equipments and many man hours, which can be seen from \cite{Ref:Drone2}. With the recent development of readily accessible drones, which are easy to use and fairly simple to operate, a revolution within the area of life guarding seems within reach. 

Utilizing the drones capability of operating and monitoring humanly inaccessible areas to control coast areas could not only result in a more safe coast environment, but also a decreased cost in regards to life saving. The use of drones in regards to emergencies and rescues has already showed promising results \cite{Ref:Drone1} \cite{Ref:Drone3} \cite{Ref:Drone4} \cite{Ref:DroneResearch1} \cite{Ref:DroneResearch1}, but mostly using humanly controlled flights.

The vision is to explore the possibilities of an easy to use and implement drone that fly autonomously within the specified area. Using computer vision to detect changes in the current scene or movement it should report back with images from the scene. The images can then be processed by a professional. As this project is considered a research project the material presented will be limited to the area of autonomous flight in good flying conditions and computer vision.\\ 
Using this approach a minimum amount of man hours is needed while the possibilities of autonomous drone flight will be utilized.\\
To ensure efficiency the group was divide into two subgroups one for the Autonomous flight and one for the Computer Vision. A graph displaying work distribution and contributions to the report can be seen in Appendix A
