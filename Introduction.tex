\chapter*{Introduction}

The present report contains the description of the  project "Using the IRIS + quadcopter for lifeguarding" developed as a part of the course  SDU-UAS-Intro 2015. 
The aim of the developed system is to enable an existing drone model to be easily configured and programmed for performing offshore patrolling with human rescue purposes. 
The required equipment, a quadcopter with a mounted gimbal with a camera, can be thus used to fly over restricted, user-specified areas at some distance from the shore. 
During its watch, if any human being is detected on this area, the drone stores the acquired images and send them on the fly to the ground station to be analyzed by the user and activate the rescue protocols. 



\section*{Acronyms}
\begin{acronym}[AWGN]
\acro{UAS}{Unmanned Aerial Systems}
\acro{ROS}{Robotics Operation System} 
\acro{UART}{Universal asynchronous receiver/transmitter} 
\end{acronym}

\section*{Motivation}
Patrolling and ensuring safety at remote and humanly inaccessible areas such as the sea has for a long time been a difficult task requiring expensive equipment and and many man hours. 
With the recent development of readily accessible drones, which are easy to use and fairly simple to operate, a revolution within the area of life guarding seems within reach. 
Utilizing the drones capability of operating and monitoring humanly inaccessible areas to control coast areas could not only result in a more safe coast environment, but also a decreased cost in regards to life saving. 
The use of drones in regards to emergencies and rescues has already showed promising results, but mostly using humanly controlled flights.
The aim of this group is to explore the possibilities of an easy to use and implement drone that fly autonomously within the specified area. 
Using vision to detect people in the current scene, the system should report back with images from the scene which can then be processed by a professional.
Using this approach a minimum amount of man hours is needed while the possibilities of autonomous drone flight will be utilized.
The idea is to deal with the problem as a sum of modules that can be developed individually by different group members and tested together, so that the maximum group efficiency can be reached while being flexible within each module.




\newpage
