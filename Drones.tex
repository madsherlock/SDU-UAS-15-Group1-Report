\chapter{Autonomous flight}
The following sections introduces the selection, set up and how autonomous flight was achieved.

\section{Drone selection}
To fulfil the vision of and easy to use and readily available drone platforms the drone selection could be narrowed down quite easily. As the primary focus of the of the project was to explore autonomous flight other concerns such as resistance to rough weather could be neglected. \\
During the course 2 drones matching the giving requirements were introduced: The dji Phantom 2 \cite{Ref:dji} and the 3DR Iris+ \cite{Ref:3dr}.\\
The obvious choice between the 2 drones was the Iris+ as it used the open source Pixhawk FCU \cite{Ref:px4}. Having an open source FCU allowed for easy interfacing and more flexibility with many tasks concerning the drone. The Iris+ drone can be seen on figure \textbf{REF}.\\

The main features and components of the Iris+ drone used for this project are:
\begin{itemize}
\item 16-22 minutes flight time.
\item Payload capacity of 400g.
\item Manual remote control.
\item Telemetry link.
\item Autonomous way point navigation.
\item Feature rich ground control software.
\end{itemize}

\section{Setup}
\subsection{Radio Controller}
The first part of the set up was to bind the Radio Controller. The receiver used is the $2.4 Ghz$ Frsky D4R-2 Reciever \cite{Ref:FrSky}. The radio transmitter and reciever can be seen on figure \textbf{REF}\\ 
To bind the controller the hood of the drone had to be taken off to access the radio receiver. Then the steps are as follows:
\begin{itemize}
\item[1.] Turn on the radio transmitter while down the button on the back of the radio transmitter.
\item[2.] Once the remote is beeping let go of the button on the back.
\item[3.] Power up the drone while holding down the F/S button on the radio receiver.
\item[4.] Release the F/S button once the radio receiver LED is flashing red and green.
\item[5.] Power off drone.
\item[6.] Power off radio transmitter.
\item[7.] If bind is done correctly the radio receiver LED should be solid green when connected to the radio transmitter and the LED blink red when data is transferred.
\end{itemize}
For further assistance the first part of the following video i suggested \url{https://www.youtube.com/watch?v=5ygCbdR4FCE}.

\subsection{Telemetry}
The telemetry used is the 3DR Radio v2 \cite{Ref:Telem} operating at $433 Mhz$. The purpose of the telemetry is to transmit and receive data from and to the ground station via the MAVLink protocol \cite{Ref:MAVLink}.\\
To connect the ground telemetry to the telemetry on the drone the 3DRRadio software was used on a windows machine. The two telemetries were connected to the PC by USB one at a time and set to the desired configuration. The configuration used can be seen on figure \textbf{REF}, where the most significant are the baud rate and the net id. When setting the configurations on the drone telemetry it is important that it is not connected to the pixhawk FCU.\\
When connected the green LED should be solid and the red LED should blink whenever data is transferred. The drone telemetry can now safely be connected to the pixhawk FCU again.

\subsection{Gimbal}
Since pictures had to be taken during flight and as these pictures had to be good enough to use computer vision on it was decided to use the Tarot T-2D Brushless Gimbal Kit \cite{Ref:Gimbal}. The gimbal adds extra weight to the drone and uses extra power from the battery to power the motors. Both of these factors leads to a significant decrease in flight time.\\
The integration and usage of the gimbal will be further explained in the vision chapter. 
\subsection{APM Planner 2.0}
- short intro etc \\

\section{Calibration}
- how and common errors\\
\section{Flight planning}
- possibilities\\
- interface\\
- syntax\\
\section{Tetst and safety}
- Safety check\\
- Dangers\\
- failed tests\\
- completed tests\\
- what was learned\\

\subsection{Mavros}
- Brief intro to ROS\\
- Nodes setup\\
- Interface\\
- Tests\\
- Future possibilities\\
\subsection{Generating search patterns}
- Mikael\\
\newpage