\chapter{Conclusion}

In section \ref{chap:autonomousflight}, autonomous flight on the Iris+ drone was explored using the APM Planner 2.0 software.
It was discovered that communicating with the drone through this software and through the Mavros ROS package,
as well as uploading a path to the drone, was achievable.
A system setup guide explaining the steps necessary to make the Iris+ drone fly autonomously has been provided in section \ref{sec:setup}.
\\
A path generation framework was developed in the form of a file format and waypoint manipulation C++ library,
and in the form of three example Matlab scripts for generating search pattern paths.
It was shown possible to design and upload custom designed search patterns to the drone via Mavros.
This gives the possibility of easily generating a flight path and launching a drone, without any advanced knowledge of drone flight,
if the processes described are integrated into a GUI application.\\

During the tests it was found that security and pre-flight checks are crucial in autonomous drone flight
as a lot of unpredictable elements are involved.
The tests also showed the excellent capabilities a relatively cheap drone has in following a path with no obstacles, even in windy conditions.\\

In section \ref{sec:vision}, is a algorithm capable of detecting humans been developed. 
The algorithm uses a skin color model to detect humans, making it robust and able to distinguish between humans and other items and lifeforms.\\
 
The skin color model, from \cite{Ref:SkinDetection}, is illumination invariant, making robust in different wheather conditions. 
The images at which the algorithm has been tested on  shows the algorithm works best on images with consistent background.
\newpage