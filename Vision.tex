\chapter{People detection}
Under this section, the solution adopted for the vision-based human detection part of the project, the arguments that led to this approach, the assumptions made, the tests performed and the obtained results are presented. 

\section{Problem definition}
Once the purpose of the vision system has been well defined, the operating conditions and the environment must be carefully studied in order to find the most suitable solution.
An embedded vision system portable by a drone, robust and fast enough and with low energy consumption is required for the task. 
The search of people in the sea entails specific problems that must be treated such as:

\begin{enumerate}[itemsep=1mm,topsep=1mm,leftmargin=.35in]
    \item False positives due to any kind of ocean animals or rocks
    \item Changes in illumination conditions due to outdoors work
    \item Forecast behavior: wind, rain
    \item Moving background and platform
    \item Highly reflective background
\end{enumerate}%

Therefore, the election of the hardware and the vision algorithm has to be taken consequently with these points.

\section{Equipment}
According to the ideas above and the available resources, a logically consistent choice seems to go through the image processing options. 
This is due to the fact that any other kind of visual system (thermal camera, structured light, time of flight measurements, etc) do not fulfill the requirements or turns out to be too expensive.
Hence, a GoPro HERO 3 camera has been chosen as a robust, light and powerful enough solution \cite{ref:HERO3}.
In order to deal with the instability and disruptions caused by the platform, the camera has been mounted on a Tarot GoPro gimbal ***(Refer. datasheet) used as a stabilizer.

\subsection{GoPro HERO 3}
The technical parameters of the chosen model have influenced the way the vision algorithm has been approached.
The characteristic small focal length of this camera model, together with the flight altitude of the drone, leads to a wide field of view, and therefore a big scanned area of ocean per frame. The high resolution of the frames and the frame rate ....blablabla

\subsection{GoPro gimbal}
Just some technical details and find datasheet


\section{Vision task}
Given the hardware platform, the image processing algorithm must deal with the remaining problems in order to reach the goal of the system. 
The initial proposals for people detection, based on movement analysis and human morphological features extraction where discarded after being subjected to more detailed studies. 
The moving platform and the waves made unfeasible a reliable solution based on movement tracking assuming that the people to be found were moving.
Furthermore, the possibility of the targets being drowning, partially or almost completely covered by water prevented any kind of solution from succeeding employing morphological analysis. 
In the view of the arguments above, the final solution was decided to perform human skin detection based on color under the assumption that at least a part of the targets would be floating. 


\subsection{Human skin identification}
In order to perform a robust detection, the RGB-H-CbCr human skin color model presented in \cite{Ref:SkinColorModel} and \cite{Ref:SkinDetection} has been implemented in the algorithm and applied for image thresholding as introduced in \cite{Ref:SkinDetection}.\\
***Details about the color model
***Details about the program here
***Equations of rules A, B and C from ref 1*****Keerthi

\subsection{Preprocessing}
The pixel-wise searching method is a very accurate but heavy process that entails a high time consumption.
In order to speed the overall performance up, some regions of interest in the image with higher possibilities of containing people must be found as a previous step. To do so, an analysis and comparison of example images of the ocean, with and without people and under different illumination conditions was carried out.  

A first approach to this consisted in filtering in the frequency domain in order to remove constant low frequencies, which would be, in this case, related to the background. Thus, only the high frequency objects would be left, containing all the isolated candidates for the detector.
Notwithstanding, it turned out that a simpler and faster threshold in the hue color space could be set up to distinguish between floating objects and the background water in all the images. 
These floating objects always involved a disruption or change in the average hue level of the whole picture, that otherwise remained constant (except for strong glare cases, which will be treated below).
However, a threshold value in hue could not be fixed, due to its dependency with the general illumination level. 
Thus, the solution adopted was to set up a dynamic threshold applying Otsu's method, as described in \cite{Ref:Otsu}.
Proceeding this way, a few small regions containing all the candidates could be input to the skin scan process, getting rid of the majority of the image with no relevant information. 

Furthermore, another assumption was made here concerning the size of the input ROIs. 
According to the hight of the drone on the flight and the focal length of the camera, an approximate estimation of the size of the pixel are of people in the image can be made. 
With it all the objects out of this boundaries can be filtered out, easing the posterior process.

As the last point on the preprocessing part, a simple and easy to implement method for dealing with the glare of the sun light on the water is suggested here, although not implemented in the project, since we came out with this problem at the very end.
The thresholding in hue was proved to fail in the cases of light reflection on small spots for certain intensities, since this glare caused also a strong enough change in the hue image histogram. To eliminate these false positives, the easiest solution would be to include a polarizer filter in the setup, which could also facilitate the color detection. 

\subsection{Postprocessing}
The result 

\section{Performance tests}
*** A database with relevant images and videos was created with testing purposes.\\
*** Computer tests\\
*** Considerations regarding onboard implementation\\
\section{Results}
\section{Conclusion}
\subsection{Further work}
Implementation of the onboard communication system to send the images on the fly
\section{Discussion}
%% Some ideas for possible sections..  We should talk about it at some point.. 
\newpage